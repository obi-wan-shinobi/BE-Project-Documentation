% !TeX program = pdflatex
% !BIB program = biber
% !TeX encoding = UTF-8
% !TeX spellcheck = en_US
\documentclass[english,utf8,biblatex]{lni}
\usepackage{booktabs}
\addbibresource{references.bib}

\begin{document}

\title[Short Title]{A Very Long Title Across Multiple Lines With Many Words and Even More Characters}
\author[Firstname Lastname \and Firstname Lastname]{%
  Firstname Lastname\footnote{University, Department, Address, Country \email{author1@university.edu}} \and
  Firstname Lastname\footnote{University, Department, Address, Country \email{author2@university.edu}}
}
\startpage{1}
\editor{Editor et al.}
\booktitle{INFORMATIK 2022 Workshops}
\yearofpublication{2022}
\lnidoi{12.34567/provided-by-the-editors}
\maketitle

\begin{abstract}
The abstract should contain between 70 und 150 words.
\end{abstract}

\begin{keywords}
LNI guidelines \and template
\end{keywords}

\section{Usage}
\Cref{fig:demo} shows a figure. \Cref{tab:demo} shows a table. Other functionality is demonstrated in \Cref{sec:demos}.

\begin{figure}
  \centering
  \includegraphics[width=.3\textwidth]{example-image}
  \caption{A demo figure}
  \label{fig:demo}
\end{figure}

\begin{table}
  \centering
  \begin{tabular}{ll}
    \toprule
    header & line \\
    \midrule
    just & some \\
    example & content \\
    \bottomrule
  \end{tabular}
  \caption{A demo table}
  \label{tab:demo}
\end{table}

The \texttt{lni} document class provides two environments for typesetting program code.

\begin{verbatim}
public class Hello {
    public static void main (String[] args) {
        System.out.println("Hello World!");
    }
}
\end{verbatim}

\begin{lstlisting}[caption={A listing with a caption}, label=L1, language=Java]
public class Hello {
    public static void main (String[] args) {
        System.out.println("Hello World!");
    }
}
\end{lstlisting}

\section{Another Section}
\label{sec:demos}
You can provide references as usual \cite{Gl01}. If you intend to use a reference as the subject or an object of a sentence, use the \texttt{\textbackslash{}citet} command, like \citet{ABC01} do.

\begin{equation}
  \lambda = 2 - \alpha
\end{equation}

A more detailed template (in German) of the \texttt{lni} document class can be found online\footnote{\url{https://www.ctan.org/pkg/lni}}.

\printbibliography

\end{document}
