\documentclass[conference]{IEEEtran}
\IEEEoverridecommandlockouts
% The preceding line is only needed to identify funding in the first footnote. If that is unneeded, please comment it out.
%\usepackage{cite}
\usepackage[authoryear]{natbib}
\usepackage{amsmath,amssymb,amsfonts}
\usepackage{algorithmic}
\usepackage{graphicx}
\usepackage{textcomp}
\usepackage{xcolor}
\def\BibTeX{{\rm B\kern-.05em{\sc i\kern-.025em b}\kern-.08em
    T\kern-.1667em\lower.7ex\hbox{E}\kern-.125emX}}
\begin{document}

\title{Astronomical Image colorization and up-scaling with generative adversarial networks\\
{\footnotesize \textsuperscript{*}Note: Sub-titles are not captured in Xplore and
should not be used}
\thanks{Identify applicable funding agency here. If none, delete this.}
}

\author{\IEEEauthorblockN{1\textsuperscript{st} Shreyas Kalvankar \IEEEauthorrefmark{1}}
\IEEEauthorblockA{\textit{Dept. of Computer Engineering} \\
\textit{K. K. Wagh Institute of Engineering Education and Research}\\
Nashik, India \\
shreyas.kalvankar@kkwagh.edu.in}
\and
\IEEEauthorblockN{2\textsuperscript{nd} Snehal Kamalapur\IEEEauthorrefmark{1}}
\IEEEauthorblockA{\textit{Dept. of Computer Engineering} \\
\textit{K. K. Wagh Institute of Engineering Education and Research}\\
Nashik, India \\
smkamalapur@kkwagh.edu.in}
\and
\IEEEauthorblockN{3\textsuperscript{rd} Hrushikesh Pandit\IEEEauthorrefmark{1}}
\IEEEauthorblockA{\textit{Dept. of Computer Engineering} \\
\textit{K. K. Wagh Institute of Engineering Education and Research}\\
Nashik, India \\
hrushikesh.pandit@kkwagh.edu.in}
\and
\IEEEauthorblockN{4\textsuperscript{th} Pranav Parwate\IEEEauthorrefmark{1}}
\IEEEauthorblockA{\textit{Dept. of Computer Engineering} \\
\textit{K. K. Wagh Institute of Engineering Education and Research}\\
Nashik, India \\
pranav.parwate@kkwagh.edu.in}
\and
\IEEEauthorblockN{5\textsuperscript{th} Atharva Patil\IEEEauthorrefmark{1}}
\IEEEauthorblockA{\textit{Dept. of Computer Engineering} \\
\textit{K. K. Wagh Institute of Engineering Education and Research}\\
Nashik, India \\
atharva.patil@kkwagh.edu.in}
}

\maketitle

\begin{abstract}
Automatic colorization of images without human intervention has been a subject of interest in the machine learning community for a brief period of time. Assigning color to an image is a highly ill-posed problem because of its innate nature of possessing very high degrees of freedom; given an image, there is often no single color-combination that is correct. Besides colorization, another problem in reconstruction of images is Single Image Super Resolution, which aims at transforming low resolution images to a higher resolution. This research aims to provide an automated approach for the problem by focusing on a very specific domain of images, namely astronomical images, and process them using Generative Adversarial Networks (GANs). We explore the usage of various models in two different color spaces, RGB and L*a*b. We use transferred learning owing to a small data set, using pre-trained ResNet-18 as a backbone, i.e. encoder for the U-net and fine-tune it further. The model produces visually appealing images which hallucinate high resolution, colorized data in these results which does not exist in the original image. We present our results by evaluating the GANs quantitatively using distance metrics such as L1 distance and L2 distance in each of the color spaces across all channels to provide a comparative analysis. We use Fréchet inception distance (FID) to compare the distribution of the generated images with the distribution of the real image to assess the model's performance.
\end{abstract}

\begin{IEEEkeywords}
Neural Networks, Generative Models, Image Colorization, Image Super-resolution
\end{IEEEkeywords}

\section{Introduction}
Automated colorization of gray scale images has been subjected to much research within the computer vision and machine learning communities. Beyond simply being fascinating from an aesthetic and artificial intelligence perspective, such capability has broad practical applications. It is an area of research that possesses great potentials in applications: from black and white photo reconstruction, image augmentation, video restoration to image enhancement for improved interpretability. 
Image downscaling is an innately lossy process. The principal objective of super resolution imaging is to reconstruct a low resolution image into a high resolution one based on a set of low-resolution images to rectify the limitations that existed while the procurement of the original low-resolution images. This is to insure better visualization and recognition for either scientific or non-scientific purposes. Even if an upscaling algorithm is particularly good, there will always be some amount of high frequency data lost from a downscale-upscale function performed on the image. Ultimately, even the best upscaling algorithms are unable to effectively reconstruct data that does not exist. Traditional methods for image upsampling rely on low-information, smooth interpolation between known pixels. Such methods can be treated as a convolution with a kernel encoding no information about the original image. A solution to the problem is by using Generative Adversarial Networks (GANs) to hallucinate high frequency data in a super scaled image that does not exist in the smaller image. Even though they increase the resolution of an image, they fail to produce the clarity desired in the super-resolution task. By using the above mentioned method, not a perfect reconstruction can be obtained albeit instead a rather plausible guess can be made at what the lost data might be, constrained to reality by a loss function penalizing deviations from the ground truth image.\\  
A huge number of raw images are present unprocessed and unnoticed in the Hubble Legacy Archives. These raw images are typically black and white, low-resolution and unfit to be shared with the world. It takes huge amounts of hours to process them. This processing is necessary because it's difficult for astronomers to distinguish objects from the raw images. Random and synthetic noise from the sensors in the telescope, changing optical characteristics in the system and noise from other bodies in the universe all make the processing further necessary. Furthermore, for the process of highlighting small features that ordinarily wouldn't be able to be picked out against noise of the image, we need colorization. The processing of the images is so time consuming that the images are rarely seen by human eyes. The problem is only likely to get worse. Not only is new data being continuously produced by Hubble Telescope, but new telescopes are soon to come online. A simplification of image processing by using artificial image colorization and super-resolution can be done in an automated fashion to make it easier for astronomers to visually identify and analyze objects in Hubble dataset.
\section{Literature Review}
\subsection{Image Colorization}
\textit{Hint Based Colorization}\\
\hspace*{0.25 in}\citet{levin2004colorization} proposed using colorization hints from the user in a quadratic cost function which imposed that neighboring pixels in space-time with similar intensities should have similar colours. This was a simple but effective method but only had hints which were provided in form of imprecise colored scribbles on the grayscale input image. But with no additional information about the image, the method was able to efficiently generate high quality colorizations. \cite{huang2005edge} addressed the color bleeding issue faced in this approach and solved it using adaptive edge detection. \cite{yatziv2006chrominance} used luminescence based weighting for hints to boost efficiency. \cite{qu2006manga} extended the original cost function to apply color continuity over similar textures along with intensities.\\ \\
\hspace*{0.1 in}\textit{Deep Colorization}\\
\hspace*{0.25 in}Owing to recent advances, the Convolutional Neural Networks are a de facto standard for solving image classification problems and their popularity continues to rise with continual improvements. CNNs are peculiar in their ability to learn and differentiate colors, patterns and shapes within an image and their ability to associate them with different classes.\\
\hspace*{0.25 in}\cite{cheng2016deep} proposed a per pixel training for neural networks using DAISY \citep{tola2008descriptor}, and semantic \citep{long2015semantic} features to predict the chrominance value for each pixel, that used bilateral filtering to smooth out accidental image artifacts. With a large enough dataset, this method proved to be superior to the example based techniques even with a simple Euclidean loss function against the ground truth values.\\
\hspace*{0.25 in}Finally, \cite{dahl2016automatic} successfully implemented a system to automatically colorize black \& white images using several ImageNet-trained layers from VGG-16 \citep{simonyan2015deep} and integrating them with auto-encoders that contained residual connections. These residual connections merged the outputs produced by the encoding VGG16 layers and the decoding portion of the network in the later stages. \cite{he2015deep} showed that deeper neural networks can be trained by reformulating layers to learn residual function with reference to layer inputs. Using this \textit{Residual Connections}, \cite{he2015deep} created the \textit{ResNets} that went as deep as 152 layers and won the 2015 ImageNet Challenge.\\ \\
\hspace*{0.1 in}\textit{Generative Adversarial Networks}\\
\hspace*{0.25 in}\cite{goodfellow2014generative} introduced the adversarial framework that provides an approach to training a neural network which uses the generative distribution of $p_g(x)$ over the input data $x$.\\
\hspace*{0.25 in}Since it's inception in 2015, many extended works of GAN have been proposed over years including DCGAN \citep{radford2016unsupervised}, Conditional-GAN \citep{mirza2014conditional}, iGAN \citep{zhu2018generative}, Pix2Pix \citep{isola2018imagetoimage}.\\
\hspace*{0.25 in}\cite{radford2016unsupervised} applied the adversarial framework for training convolutional neural networks as generative models for images, demonstrating the viability of \textit{deep convolutional generative adversarial networks}.\\
\hspace*{0.25 in}DCGAN is the standard architecture to generate images from random noise. Instead of generating images from random noise, Conditional-GAN \citep{mirza2014conditional} uses a condition to generate output image. For e.g. a grayscale image is the condition for colorization of image. Pix2Pix \citep{isola2018imagetoimage} is a Conditional-GAN with images as the conditions. The network can learn a mapping from input image to output image and also learn a separate loss function to train this mapping. Pix2Pix is considered to be the state of the art architecture for image-image translation problems like colorization.
\subsection{Image Upscaling}
\textit{Frequency-domain-based SR image approach}\\
\hspace*{0.25 in} \cite{tsai1984multiframe} proposed the frequency domain SR method, where SR computation was considered for the noise free low resolution images. They transformed the low resolution images into Discrete Fourier transform (DFT) and further combined it as per the relationship between the aliased DFT coefficient of the observed low resolution image and that of unknown high resolution image. Then the output is transformed back into the spatial domain where a higher resolution is now achieved.\\
\hspace*{0.25 in} While Frequency-domain-based SR extrapolates high frequeny information from the low resolution images and is thus useful, however they fall short in real world applications.\\ \\
\hspace*{0.1 in}\textit{The interpolation based SR image approach}\\
\hspace*{0.25 in} The interpolation-based SR approach constructs a high resolution image by casting all the low resolution images to the reference image and then combining all the information available from every image available.
The method consists of the following three stages
(i) the registration stage for aligning the low-resolution input images,
(ii) the interpolation stage for producing a higher-resolution image, and
(iii) the deblurring stage which enhances the
reconstructed high-resolution image produced in the step (ii).

However, as each low resolution image adds a few new details before finally deblurring them, this method cannot be used if only a single reference image is available.\\ \\
\hspace*{0.1 in}\textit{Regularization-based SR image approach}\\
\hspace*{0.25 in} Most known Bayesian-based SR approaches are maximum likelihood (ML) estimation approach  and maximum a posterior (MAP) estimation approach.\\
\hspace*{0.25 in}  While \cite{Brian1996ML} proposed the first ML estimation based SR approach with the aim to find the ML estimation of high resolution image, some proposed a MAP estimation approach. MAP SR tries to takes into consideration the prior image model to reflect the expectation of the unknown high resolution image.\\ \\
\hspace*{0.1 in}\textit{Super Resolution - Generative Adversarial Networks (SR-GAN)}\\
\hspace*{0.25 in} The Genrative Adversarial Network \citep{goodfellow2014generative}, has two neural networks, the Generator and the Discriminator. These networks compete with each other in a zero-sum game.
\cite{ledig2017photorealistic} introduced SRGAN in 2017, which used a SRResNet to upscale images with an upscaling factor of 4x. SRGAN is currently the state of the art on public benchmark datasets.

\section{Prepare Your Paper Before Styling}
Before you begin to format your paper, first write and save the content as a 
separate text file. Complete all content and organizational editing before 
formatting. Please note sections \ref{AA}--\ref{SCM} below for more information on 
proofreading, spelling and grammar.

Keep your text and graphic files separate until after the text has been 
formatted and styled. Do not number text heads---{\LaTeX} will do that 
for you.

\subsection{Abbreviations and Acronyms}\label{AA}
Define abbreviations and acronyms the first time they are used in the text, 
even after they have been defined in the abstract. Abbreviations such as 
IEEE, SI, MKS, CGS, ac, dc, and rms do not have to be defined. Do not use 
abbreviations in the title or heads unless they are unavoidable.

\subsection{Units}
\begin{itemize}
\item Use either SI (MKS) or CGS as primary units. (SI units are encouraged.) English units may be used as secondary units (in parentheses). An exception would be the use of English units as identifiers in trade, such as ``3.5-inch disk drive''.
\item Avoid combining SI and CGS units, such as current in amperes and magnetic field in oersteds. This often leads to confusion because equations do not balance dimensionally. If you must use mixed units, clearly state the units for each quantity that you use in an equation.
\item Do not mix complete spellings and abbreviations of units: ``Wb/m\textsuperscript{2}'' or ``webers per square meter'', not ``webers/m\textsuperscript{2}''. Spell out units when they appear in text: ``. . . a few henries'', not ``. . . a few H''.
\item Use a zero before decimal points: ``0.25'', not ``.25''. Use ``cm\textsuperscript{3}'', not ``cc''.)
\end{itemize}

\section{Dataset}


\subsection{Equations}
Number equations consecutively. To make your 
equations more compact, you may use the solidus (~/~), the exp function, or 
appropriate exponents. Italicize Roman symbols for quantities and variables, 
but not Greek symbols. Use a long dash rather than a hyphen for a minus 
sign. Punctuate equations with commas or periods when they are part of a 
sentence, as in:
\begin{equation}
a+b=\gamma\label{eq}
\end{equation}

Be sure that the 
symbols in your equation have been defined before or immediately following 
the equation. Use ``\eqref{eq}'', not ``Eq.~\eqref{eq}'' or ``equation \eqref{eq}'', except at 
the beginning of a sentence: ``Equation \eqref{eq} is . . .''

\subsection{\LaTeX-Specific Advice}

Please use ``soft'' (e.g., \verb|\eqref{Eq}|) cross references instead
of ``hard'' references (e.g., \verb|(1)|). That will make it possible
to combine sections, add equations, or change the order of figures or
citations without having to go through the file line by line.

Please don't use the \verb|{eqnarray}| equation environment. Use
\verb|{align}| or \verb|{IEEEeqnarray}| instead. The \verb|{eqnarray}|
environment leaves unsightly spaces around relation symbols.

Please note that the \verb|{subequations}| environment in {\LaTeX}
will increment the main equation counter even when there are no
equation numbers displayed. If you forget that, you might write an
article in which the equation numbers skip from (17) to (20), causing
the copy editors to wonder if you've discovered a new method of
counting.

{\BibTeX} does not work by magic. It doesn't get the bibliographic
data from thin air but from .bib files. If you use {\BibTeX} to produce a
bibliography you must send the .bib files. 

{\LaTeX} can't read your mind. If you assign the same label to a
subsubsection and a table, you might find that Table I has been cross
referenced as Table IV-B3. 

{\LaTeX} does not have precognitive abilities. If you put a
\verb|\label| command before the command that updates the counter it's
supposed to be using, the label will pick up the last counter to be
cross referenced instead. In particular, a \verb|\label| command
should not go before the caption of a figure or a table.

Do not use \verb|\nonumber| inside the \verb|{array}| environment. It
will not stop equation numbers inside \verb|{array}| (there won't be
any anyway) and it might stop a wanted equation number in the
surrounding equation.

\subsection{Some Common Mistakes}\label{SCM}
\begin{itemize}
\item The word ``data'' is plural, not singular.
\item The subscript for the permeability of vacuum $\mu_{0}$, and other common scientific constants, is zero with subscript formatting, not a lowercase letter ``o''.
\item In American English, commas, semicolons, periods, question and exclamation marks are located within quotation marks only when a complete thought or name is cited, such as a title or full quotation. When quotation marks are used, instead of a bold or italic typeface, to highlight a word or phrase, punctuation should appear outside of the quotation marks. A parenthetical phrase or statement at the end of a sentence is punctuated outside of the closing parenthesis (like this). (A parenthetical sentence is punctuated within the parentheses.)
\item A graph within a graph is an ``inset'', not an ``insert''. The word alternatively is preferred to the word ``alternately'' (unless you really mean something that alternates).
\item Do not use the word ``essentially'' to mean ``approximately'' or ``effectively''.
\item In your paper title, if the words ``that uses'' can accurately replace the word ``using'', capitalize the ``u''; if not, keep using lower-cased.
\item Be aware of the different meanings of the homophones ``affect'' and ``effect'', ``complement'' and ``compliment'', ``discreet'' and ``discrete'', ``principal'' and ``principle''.
\item Do not confuse ``imply'' and ``infer''.
\item The prefix ``non'' is not a word; it should be joined to the word it modifies, usually without a hyphen.
\item There is no period after the ``et'' in the Latin abbreviation ``et al.''.
\item The abbreviation ``i.e.'' means ``that is'', and the abbreviation ``e.g.'' means ``for example''.
\end{itemize}
An excellent style manual for science writers is \cite{cheng2016deep}.

\subsection{Authors and Affiliations}
\textbf{The class file is designed for, but not limited to, six authors.} A 
minimum of one author is required for all conference articles. Author names 
should be listed starting from left to right and then moving down to the 
next line. This is the author sequence that will be used in future citations 
and by indexing services. Names should not be listed in columns nor group by 
affiliation. Please keep your affiliations as succinct as possible (for 
example, do not differentiate among departments of the same organization).

\subsection{Identify the Headings}
Headings, or heads, are organizational devices that guide the reader through 
your paper. There are two types: component heads and text heads.

Component heads identify the different components of your paper and are not 
topically subordinate to each other. Examples include Acknowledgments and 
References and, for these, the correct style to use is ``Heading 5''. Use 
``figure caption'' for your Figure captions, and ``table head'' for your 
table title. Run-in heads, such as ``Abstract'', will require you to apply a 
style (in this case, italic) in addition to the style provided by the drop 
down menu to differentiate the head from the text.

Text heads organize the topics on a relational, hierarchical basis. For 
example, the paper title is the primary text head because all subsequent 
material relates and elaborates on this one topic. If there are two or more 
sub-topics, the next level head (uppercase Roman numerals) should be used 
and, conversely, if there are not at least two sub-topics, then no subheads 
should be introduced.

\subsection{Figures and Tables}
\paragraph{Positioning Figures and Tables} Place figures and tables at the top and 
bottom of columns. Avoid placing them in the middle of columns. Large 
figures and tables may span across both columns. Figure captions should be 
below the figures; table heads should appear above the tables. Insert 
figures and tables after they are cited in the text. Use the abbreviation 
``Fig.~\ref{fig}'', even at the beginning of a sentence.

\begin{table}[htbp]
\caption{Table Type Styles}
\begin{center}
\begin{tabular}{|c|c|c|c|}
\hline
\textbf{Table}&\multicolumn{3}{|c|}{\textbf{Table Column Head}} \\
\cline{2-4} 
\textbf{Head} & \textbf{\textit{Table column subhead}}& \textbf{\textit{Subhead}}& \textbf{\textit{Subhead}} \\
\hline
copy& More table copy$^{\mathrm{a}}$& &  \\
\hline
\multicolumn{4}{l}{$^{\mathrm{a}}$Sample of a Table footnote.}
\end{tabular}
\label{tab1}
\end{center}
\end{table}

\begin{figure}[htbp]
\centerline{\includegraphics{fig1.png}}
\caption{Example of a figure caption.}
\label{fig}
\end{figure}

Figure Labels: Use 8 point Times New Roman for Figure labels. Use words 
rather than symbols or abbreviations when writing Figure axis labels to 
avoid confusing the reader. As an example, write the quantity 
``Magnetization'', or ``Magnetization, M'', not just ``M''. If including 
units in the label, present them within parentheses. Do not label axes only 
with units. In the example, write ``Magnetization (A/m)'' or ``Magnetization 
\{A[m(1)]\}'', not just ``A/m''. Do not label axes with a ratio of 
quantities and units. For example, write ``Temperature (K)'', not 
``Temperature/K''.

\section*{Acknowledgment}

The preferred spelling of the word ``acknowledgment'' in America is without 
an ``e'' after the ``g''. Avoid the stilted expression ``one of us (R. B. 
G.) thanks $\ldots$''. Instead, try ``R. B. G. thanks$\ldots$''. Put sponsor 
acknowledgments in the unnumbered footnote on the first page.

\section*{References}

Please number citations consecutively within brackets \cite{b1}. The 
sentence punctuation follows the bracket \cite{b2}. Refer simply to the reference 
number, as in \cite{b3}---do not use ``Ref. \cite{b3}'' or ``reference \cite{b3}'' except at 
the beginning of a sentence: ``Reference \cite{b3} was the first $\ldots$''

Number footnotes separately in superscripts. Place the actual footnote at 
the bottom of the column in which it was cited. Do not put footnotes in the 
abstract or reference list. Use letters for table footnotes.

Unless there are six authors or more give all authors' names; do not use 
``et al.''. Papers that have not been published, even if they have been 
submitted for publication, should be cited as ``unpublished'' \cite{b4}. Papers 
that have been accepted for publication should be cited as ``in press'' \cite{b5}. 
Capitalize only the first word in a paper title, except for proper nouns and 
element symbols.

For papers published in translation journals, please give the English 
citation first, followed by the original foreign-language citation \cite{b6}.

\bibliographystyle{dcu}
\bibliography{biblo.bib}
% \begin{thebibliography}{00}
% \bibitem{b1} G. Eason, B. Noble, and I. N. Sneddon, ``On certain integrals of Lipschitz-Hankel type involving products of Bessel functions,'' Phil. Trans. Roy. Soc. London, vol. A247, pp. 529--551, April 1955.
% \bibitem{b2} J. Clerk Maxwell, A Treatise on Electricity and Magnetism, 3rd ed., vol. 2. Oxford: Clarendon, 1892, pp.68--73.
% \bibitem{b3} I. S. Jacobs and C. P. Bean, ``Fine particles, thin films and exchange anisotropy,'' in Magnetism, vol. III, G. T. Rado and H. Suhl, Eds. New York: Academic, 1963, pp. 271--350.
% \bibitem{b4} K. Elissa, ``Title of paper if known,'' unpublished.
% \bibitem{b5} R. Nicole, ``Title of paper with only first word capitalized,'' J. Name Stand. Abbrev., in press.
% \bibitem{b6} Y. Yorozu, M. Hirano, K. Oka, and Y. Tagawa, ``Electron spectroscopy studies on magneto-optical media and plastic substrate interface,'' IEEE Transl. J. Magn. Japan, vol. 2, pp. 740--741, August 1987 [Digests 9th Annual Conf. Magnetics Japan, p. 301, 1982].
% \bibitem{b7} M. Young, The Technical Writer's Handbook. Mill Valley, CA: University Science, 1989.
% \end{thebibliography}
\vspace{12pt}
\color{red}
IEEE conference templates contain guidance text for composing and formatting conference papers. Please ensure that all template text is removed from your conference paper prior to submission to the conference. Failure to remove the template text from your paper may result in your paper not being published.

\end{document}
